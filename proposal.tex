\documentclass[12pt]{article}

% Any percent sign marks a comment to the end of the line

% Every latex document starts with a documentclass declaration like this
% The option dvips allows for graphics, 12pt is the font size, and article
%   is the style


\usepackage{url}

% These are additional packages for "pdflatex", graphics, and to include
% hyperlinks inside a document.

\setlength{\oddsidemargin}{0.25in}
\setlength{\textwidth}{6.5in}
\setlength{\topmargin}{0in}
\setlength{\textheight}{8.5in}

% These force using more of the margins that is the default style

\begin{document}

% Everything after this becomes content
% Replace the text between curly brackets with your own

\title{Team 07 - Project Proposal \\
	   \large CS6220 - Data Mining Techniques - Fall 2017 \\
	   \large Northeastern University}
\author{Nakul Camasamudram, Rosy Parmar, Rahul Verma, Guiheng Zhou}
\date{\today}

% You can leave out "date" and it will be added automatically for today
% You can change the "\today" date to any text you like


\maketitle

% This command causes the title to be created in the document

\section{The Dataset}

% An article style is separated into sections and subsections with 
%   markup such as this.  Use \section*{Principles} for unnumbered sections.

Our team will be using "The Instacart Online Grocery Shopping Dataset 2017". Instacart is an American company that operates as a same-day grocery delivery service.\cite{instacartwiki} This anonymized dataset contains a sample of over 3 million grocery orders from more than 200,000 Instacart users. For each user, the dataset has 4 to 100 of their orders, with the sequence of products purchased in each order. The week and hour of the day the order was placed, and a relative measure of time between orders is also available.\cite{instacartblogpost2017}.

\textbf{File Descriptions:}\cite{instacartkaggle}
Each entity (customer, product, order, aisle, etc.) has an associated unique id.
\begin{itemize}
	\item \textbf{aisles.csv}
		\begin{verbatim}
		 aisle_id,aisle  
		 1,prepared soups salads  
		 2,specialty cheeses  
		 3,energy granola bars  
		 ...
		\end{verbatim}
	\item \textbf{departments.csv}
		\begin{verbatim}
			 department_id,department  
			 1,frozen  
			 2,other  
			 3,bakery  
			 ...	
		\end{verbatim}
	\item \textbf{order\_products\_\_*.csv}: These files specify which products were purchased in each order. "order\_products\_\_prior.csv" contains previous orders for all customers. "reordered" indicates that the customer has a previous order that contains the product.
	
		\begin{verbatim}
			 order_id,product_id,add_to_cart_order,reordered  
			 1,49302,1,1  
			 1,11109,2,1  
			 1,10246,3,0  
			 ... 
		\end{verbatim}
	\item \textbf{orders.csv}: This file tells to which set (prior, train, test) an order belongs.
		\begin{verbatim}order_id,user_id,eval_set,order_number,order_dow,order_hour_of_day,
		    days_since_prior_order  
			2539329,1,prior,1,2,08,	  
			2398795,1,prior,2,3,07,15.0  
			473747,1,prior,3,3,12,21.0  
			...	
		\end{verbatim}

	\item \textbf{products.csv}
		\begin{verbatim}
			 product_id,product_name,aisle_id,department_id
			 1,Chocolate Sandwich Cookies,61,19  
			 2,All-Seasons Salt,104,13  
			 3,Robust Golden Unsweetened Oolong Tea,94,7  
			 ...
		\end{verbatim}
\end{itemize}

\section{Questions to be answered}
Our goal is to create a product recommendation system using the Instacart data that would answer the following questions
\begin{itemize}
\item Given a set of products in a customer's basket, what is another associated set of products he/she is likely to buy?
\item Given a customer, what products could be recommended to him/her so that a purchase would be made?
\end{itemize}

\section{Algorithms}
We plan on applying a subset of the below algorithms to explore relationships in the dataset and answer the above mentioned questions.

\begin{itemize}
	\item Association rule learning: Apriori Algorithm, FP Growth
	\item Recommender systems: Collaborative filtering
	\item Other association exploration: Clustering, Word2vec 
\end{itemize}
 
\section{Division of work}
\begin{itemize}
	\item \textbf{Nakul}: 
Documentation, Domain Research, Algorithm Research and Usage, Find basic stats and answers, Attribute research and selections, Dimension Reduction
	\item \textbf{Rosy}:
Documentation, Algorithm Research and Usage, Research and experimentation of libraries, Data cleaning
	\item \textbf{Rahul}:
Documentation, Algorithm Research and Usage, Find basic stats and answers 
	\item \textbf{Guiheng}:
Documentation, Algorithm Research and Usage, Research and experimentation of libraries, Data cleaning

\end{itemize}

\begin{thebibliography}{99}

\bibitem{instacartwiki} https://scholar.google.com/citations?hl=en\&user=09kJn28AAAAJ&view_op=list_works&sortby=title

\bibitem{instacartblogpost2017} Stanley, Jeremy. "3 Million Instacart Orders, Open Sourced – tech-at-instacart." Tech-at-instacart. May 03, 2017. Accessed October 11, 2017. https://tech.instacart.com/3-million-instacart-orders-open-sourced-d40d29ead6f2.
\bibitem{instacartkaggle}Instacart. "Instacart Market Basket Analysis - Data." Instacart Market Basket Analysis. Accessed October 11, 2017. https://www.kaggle.com/c/instacart-market-basket-analysis/data.
\end{thebibliography}



\end{document}